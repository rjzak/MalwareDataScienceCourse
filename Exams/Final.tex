\documentclass[letter,10pt]{article}
\usepackage[letterpaper, margin=0.75in]{geometry}
\geometry{letterpaper}
\usepackage{bbding} % For checkmark in itemize
\newcommand*\tick{\item[\Checkmark]}
\newcommand*\fail{\item[\XSolidBrush]}

\begin{document}

\begin{enumerate}
    \item Which common Linux program can be used to identify the type for a given file?
    \begin{itemize}
        \tick file
        \fail type
        \fail uname
        \fail which
        \fail whoami
    \end{itemize}
    
    \item Some file formats are executable, some are not. Besides PE32, which is another executable file format?
    \begin{itemize}
        \tick Mach-O
        \fail Fortran
        \fail Markdown
        \fail Common Document Format
    \end{itemize}
    
    \item Entry points apply to which file formats? (Select all that apply)
    \begin{itemize}
        \tick PE32
        \tick MACH-O
        \fail PDF
        \fail RTF
        \fail .DOCX
    \end{itemize}
    
    \item Which is NOT a feature or use case for a packer?
    \begin{itemize}
        \tick Improves operating system compatibility
        \fail Obfuscation
        \fail Smaller file size
        \fail Protects intellectual property
        \fail Prevent reverse engineering efforts
    \end{itemize}
    
    \item Which is NOT a way to safely handle malware?
    \begin{itemize}
        \tick Rename the file
        \fail Encrypt or compress the file
        \fail Use a different operating system
        \fail Use a different CPU architecture
        \fail Disable execution on a partition or directory
    \end{itemize}
    
    \item What is contained in the Rich Header?
    \begin{itemize}
        \tick Additional compiler/linker information
        \fail Name of the author
        \fail Number of sections
        \fail If the file is part of the operating system
    \end{itemize}
    
    \item After a good model is created, your job of making a good malware detection model is done.
    \begin{itemize}
        \tick False
        \fail True
    \end{itemize}
    
    \item If the \texttt{file} command doesn't specify a packer, does it mean the file is unpacked?
    \begin{itemize}
        \tick False
        \fail True
    \end{itemize}
    
    \item What might be the entropy level for an unpacked executable?
    \begin{itemize}
        \tick 6.5
        \fail 7.5
        \fail 8.5
        \fail 9.5
    \end{itemize}
    
    \item What might be the entropy level for a packed executable?
    \begin{itemize}
        \fail 6.5
        \tick 7.5
        \fail 8.5
        \fail 9.5
    \end{itemize}
    
    \item What is the most important area for consideration when making a model?
    \begin{itemize}
        \tick Ensuring the data is accurate and is a representative sample.
        \fail Using the Linux operating system.
        \fail Write code using Python.
        \fail Making use of GPU acceleration, or else training takes too long.
    \end{itemize}
    
    \item What might be a way to ensure the sample is diverse?
    \begin{itemize}
        \tick Use a similarity hash to make sure the files aren't alike.
        \tick Use AV Class to make sure there are a lot of different malware families in the dataset.
        \fail Make sure the SHA-1 or MD-5 hashes are different.
        \fail Take all the data available, and randomly select files from the collection for the dataset.
        \fail Trick question, this isn't an issue to consider.
    \end{itemize}
    
    \item Identify the algorithms which are similarity hashes (Select all that apply)
    \begin{itemize}
        \tick SSDeep
        \tick SDHash
        \tick LZJD
        \fail MD-5
        \fail SHA-1
        \fail RIPEMD
        \fail Whirlpool
    \end{itemize}
    
    \item It's a good idea to just use executables from \texttt{C:\textbackslash Windows} and \texttt{C:\textbackslash Program Files} for goodware samples.
    \begin{itemize}
        \tick False
        \fail True
    \end{itemize}
    
    \item What technique can be used to have the model select the best features?
    \begin{itemize}
        \tick Regularization (ElasticNet)
        \fail Neural Networks
        \fail Low-entropy detection
        \fail False Positives
    \end{itemize}
    
    \item When would you use the libsvm file format for your dataset?
    \begin{itemize}
        \tick When the data is sparse
        \fail When analyzing PDF malware
        \fail When using entropy as a feature
        \fail When using the \texttt{pefile} Python module
    \end{itemize}
    
    \item Define malware.
    \begin{itemize}
        \tick Malicious \textit{intent} against computer \textit{owner}.
    \end{itemize}
    
    \item Define conceptual drift.
    \begin{itemize}
        \tick Over time, the malware changes, the malware authors change, so the features used, and the relationship between the features, will also change.
    \end{itemize}
    
    \item When would you want to use dynamic analysis over static analysis?
    \begin{itemize}
        \tick When the malware is packed and/or encrypted.
        \tick Malware is complete, it simply downloads another part which does further malicious behavior.
    \end{itemize}
    
    \item Why would it be acceptable to use AUC instead of accuracy as an acceptable metric for model performance?
    \begin{itemize}
        \tick A very high AUC means the accuracy was high too.
        \tick A high AUC shows that the model was able to accurate sort the files by malicious vs. benign.
    \end{itemize}
    
    \item For our purposes, why shouldn't a tool which might be used by a hacker, such as a port scanner, be considered malware?
    \begin{itemize}
        \tick We can't know the \textit{intent} of the user of the tool, and the author of the tool didn't \textit{intend} for malicious usage.
    \end{itemize}
    
    \item When might it be too difficult, or impossible, to create a Yara rule for a given malware family?
    \begin{itemize}
        \tick Malware family used different packers, or is otherwise not part of common source code.
        \tick Malware which is supposedly related, but having different characteristics, used for different things, but maybe attributed to a common attack, APT, or threat actor.
    \end{itemize}
    
\end{enumerate}

\end{document}